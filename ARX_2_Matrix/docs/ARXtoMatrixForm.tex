\documentclass{article}
\usepackage{blindtext}
\usepackage[utf8]{inputenc}
\usepackage{amsmath}
\usepackage{array}
\usepackage{textcomp}




\newenvironment{conditions}
{\par\vspace{\abovedisplayskip}\noindent\begin{tabular}{>{$}l<{$} @{${}={}$} l}}
{\end{tabular}\par\vspace{\belowdisplayskip}}




\begin{document}



\section{ARX to Matrix form}
Starting from $ARX$
\begin{equation*}
  A(q)y(t) = B(q)u(t-n_k) +e(t)
\end{equation*}

$q$ is delay operator. Specifically,
\begin{align*}
&A(q) = 1 + a_1q^{-1} + ... + a_{n_a}q^{-n_a}\\
&B(q) = b_1 + b_2q^{-1} + ... + b_{n_b}q^{-n_b+1}
\end{align*}

Let's assume $n_a = n_b = p$, $p$ is $ARX$ order. And $n_k = 0$
\begin{align*}
&A(q) = 1 + a_1q^{-1} + ... + a_{p}q^{-p}\\
&B(q) = b_1 + b_2q^{-1} + ... + b_{p}q^{-p+1}
\end{align*}
\\
\\

Convert $ARX$ form to Matrix form.
This transformation makes calculating $y(t)$ easily.

\begin{align*}
	&y(t) + a_1y(t-1) + ... + a_{p}y(t-p) = b_1u(t) + ... + b_{p}u(t-p+1)\\\\
	&y(t) = Ay_p(t-p) + Bu_p(t-p+1)
\end{align*}

\begin{equation*}
	y_p(t-p) =
	\begin{bmatrix}
		y(t-p)\\
		y(t-p+1)\\
		\vdots\\
		y(t-1)
	\end{bmatrix}_{p\times1}
	,u_p(t-p+1) = 
	\begin{bmatrix}
		u(t-p+1)\\
		u(t-p+2)\\
		\vdots\\
		u(t)
	\end{bmatrix}_{p\times1}
\end{equation*}
\\
\\
Divide $u$ to $u$ and $x$. $u$ is control inputs, and $x$ is system inputs.

\begin{equation}\label{eq:1}
	y(t) = Ay_p(t-p) + B_{1}u_p(t-p+1) + B_{2}x_p(t-p+1)
\end{equation}
\\
\\

Purpose of MPC is predict model's optimum output in prediction horizon. 
When length of prediction horizon is $h$, calculating this value is needed to optimization process in MPC.

\begin{equation}\label{eq:2}
	y_h(t+1) =
	\begin{bmatrix}
		y(t+1)\\
		y(t+2)\\
		\vdots\\
		y(t+h)
	\end{bmatrix}
\end{equation}

To get the elements of \eqref{eq:2}, equation \eqref{eq:1} can be used.
\begin{align*}
	y(t) &= Ay_p(t-p) + B_{1}u_p(t-p+1) + B_{2}x_p(t-p+1)\\
	y(t+1) &= Ay_p(t-p+1) + B_{1}u_p(t-p+2) + B_{2}x_p(t-p+2)\\
	& \vdots\\
	y(t+p) &= Ay_p(t) + B_{1}u_p(t+1) + B_{2}x_p(t+1)\\
	& \vdots\\
	y(t+n) &= Ay_p(t-p+n) + B_1u_p(t-p+1+n) + B_2x_p(t-p+1+n)
\end{align*}
\\\\

Calculating $y_p(t+1)$ with $y_p(t)$ can simplify equation \eqref{eq:1}.

\begin{align*}
\begin{split}
	\begin{bmatrix}
		y(t+1)\\
		y(t+2)\\
		\vdots\\
		y(t+p)
	\end{bmatrix}
	=
	\begin{bmatrix}
	\begin{bmatrix}
		0 & 1 & \cdots & 0\\
		\vdots & \vdots & \ddots & \vdots\\ 
		0 & 0 & \cdots & 1\\
	\end{bmatrix}  \\ 
	A
	\end{bmatrix}_{p\times p}
	\begin{bmatrix}
	y(t)\\
	y(t+1)\\
	\vdots\\
	y(t+p-1)
	\end{bmatrix}
	& +
	\begin{bmatrix}
	\begin{bmatrix}
	0 & \cdots & 0\\
	\vdots & \ddots & \vdots\\ 
	0 & \cdots & 0\\
	\end{bmatrix}\\
	B_1
	\end{bmatrix}_{p\times p}
	\begin{bmatrix}
	u(t+1-d_1)\\
	u(t+2-d_1)\\
	\vdots\\
	u(t+p-d_1)
	\end{bmatrix}\\
	& 	+
	\begin{bmatrix}
	\begin{bmatrix}
	0 & \cdots & 0\\
	\vdots & \ddots & \vdots\\ 
	0 & \cdots & 0\\
	\end{bmatrix}\\
	B_2
	\end{bmatrix}_{p\times p}
	\begin{bmatrix}
	x(t+1-d_2)\\
	x(t+2-d_2)\\
	\vdots\\
	x(t+p-d_2)
	\end{bmatrix}\\
\end{split}
\end{align*}

\begin{equation} \label{eq:3}
\\
	y_p(t+1) = Cy_p(t) + Du_p(t+1-d_1) + Ex_p(t+1-d_2)
\end{equation}
\\

$d_1$ and $d_2$ is delay of control inputs and system inputs. From equation \eqref{eq:1} and \eqref{eq:3}, $y(t+n)$ can be written with $y_p(t-p)$. It means that future output($y(t+n)$) can be composed of past information(before time step $t$). 


For simplicity, let's assume $d_1 = d_2 = 1$.

\begin{align*}
	y(t+n) &= Ay_p(t+n-p) + B_1u_p(t+n-p) + B_2x_p(t+n-p)\\
	&= A\{Cy_p(t+n-p-1) + Du_p(t+n-p-1) + Ex_p(t+n-p-1)\}\\
	&+ B_1u_p(t+n-p) + B_2x_p(t+n-p)\\
	&= ACy_p(t+n-p-1) + ADu_p(t+n-p-1) + AEx_p(t+n-p-1)\\
	&+ B_1u_p(t+n-p) + B_2x_p(t+n-p)\\\\
	&=AC\{Cy_p(t+n-p-2) + Du_p(t+n-p-2) + Ex_p(t+n-p-2)\}\\
	& + ADu_p(t+n-p-1) + AEx_p(t+n-p-1) + B_1u_p(t+n-p) 
	+ B_2x_p(t+n-p)\\
	&=AC^2y_p(t+n-p-2) + ACDu_p(t+n-p-2) + ACEx_p(t+n-p-2)\\
	& + ADu_p(t+n-p-1) + AEx_p(t+n-p-1) + B_1u_p(t+n-p) 
	+ B_2x_p(t+n-p)\\
	& \vdots\\
	&= AC^{n-1}y_p(t-p+1)\\
	&+ \sum_{i=1}^{n-1} AC^{n-(i+1)}Du_p(t+n-p+(i-n)) + B_1u_p(t+n-p)\\
	&+ \sum_{i=1}^{n-1} AC^{n-(i+1)}Ex_p(t+n-p+(i-n))+ B_2x_p(t+n-p)
\end{align*}




With this equation, compute the outputs in prediction horizon.($y(t+1) ~ y(t+h)$)
 In next session, matrix $T$ and $U$ will come out. Deriving that matrix will be omit... :)

\subsection{Derive $T$ and $U$.}

Our objective is calculating $y_h(t+1)$. First, let's look at $u_p$ term in $y_h(t+1)$.
\begin{align*}
	y(t+1) & =\;  \sim + \:B_1u_p(t+1-p)\\
	y(t+2) & =\;  \sim + \:B_1u_p(t+2-p)\\
	y(t+3) & =\;  \sim + \:B_1u_p(t+3-p)\\
		   & \vdots\\
	y(t+h-1) &= \;  \sim + \:B_1u_p(t+h-1-p)\\
	y(t+h) &= \;  \sim + \:B_1u_p(t+h-p)\\
\end{align*}

can be summarized like this
\begin{equation*}
	y_h(t+1) =\;  \sim + \: \overline{B_1}u_{p+h}(t+1-p)
\end{equation*}

in the same manner, 

\begin{align*}
	y(t+1) & =\;  \sim + \:ADu_p(t+1-p)\\
	y(t+2) & =\;  \sim + \:ADu_p(t+2-p)\\
	y(t+3) & =\;  \sim + \:ADu_p(t+3-p)\\
		   & \vdots\\
	y(t+h-1) &= \;  \sim + \:ADu_p(t+h-1-p)\\
	y(t+h) &= \;  \sim + \:ADu_p(t+h-p)\\
\end{align*}

can be summarized like this
\begin{equation*}
	y_h(t+1) =\;  \sim + \: \overline{AD}u_{p+h}(t+1-p)
\end{equation*}


at the end..\\

\begin{align*}
	y(t+1) & =\;  \sim + \:AC^{n-2}Du_p(t+1-p)\\
	y(t+2) & =\;  \sim + \:AC^{n-2}Du_p(t+2-p)\\
	y(t+3) & =\;  \sim + \:AC^{n-2}Du_p(t+3-p)\\
		   & \vdots\\
	y(t+h-1) &= \;  \sim + \:AC^{n-2}Du_p(t+h-1-p)\\
	y(t+h) &= \;  \sim + \:AC^{n-2}Du_p(t+h-p)\\
\end{align*}

can be summarized like this
\begin{equation*}
	y_h(t+1) =\;  \sim + \: \overline{AC^{n-2}D}u_{p+h}(t+1-p)
\end{equation*}
\\

So coefficient of $u_p$ term can be written as $T$. 
\begin{equation*}
	T = \: \overline{B_1} +  \overline{AD} + \hdots +  \overline{AC^{n-2}D}
\end{equation*}
\\

In the same manner, coeffiecient of $x_p$ term can be written as $U$. 
\begin{equation*}
	U = \: \overline{B_2} +  \overline{AD} + \hdots +  \overline{AC^{n-2}E}
\end{equation*}
\\

Coefficient of $y_p$ term is sum of $AC^{n}$. Let's write that coefficients as $V$. Then 


\begin{align*}
	&y_h(t+1) = Tu_{p+h-1}(t-p+1) + Ux_{p+h-1}(t-p+1) + Vy_p(t-p+1)\\\\
	=&\begin{bmatrix}
		y(t+1)\\
		y(t+2)\\
		\vdots\\
		y(t+h)
	\end{bmatrix}
	= 
	\begin{bmatrix}
	T_1\\
	T_2
	\end{bmatrix}^T
	\begin{bmatrix}
		u(t-p+1)\\
		u(t-p+2)\\
		\vdots\\
		u(t-1)\\
		u(t)\\
		\hline
		u(t+1)\\
		\vdots\\
		u(t+h-1)
	\end{bmatrix}
	+
	\cdots
\end{align*}



\begin{align*}
	=y_h(t+1) = T_1u_p(t-p+1)+ T_2u_{h-1}(t+1) + Ux_{p+h-1}(t-p+1) + Vy_p(t-p+1)
\end{align*}
\\
\begin{conditions}
y_h(t+1)	&	calculated outputs($T_in$ or $Gas$)\\
u_p(t-p+1) & past control inputs\\
u_{h-1}(t+1) & future control inputs - calculated with optimization\\
x_{p+h-1}(t-p+1) & past and future system inputs\\
y_p(t-p+1) & past outputs
\end{conditions}

except $y_h(t+1)$ and $u_{h-1}(t+1)$, other variables comes from dataset.
To check optimization process and outcomes comfortably, matrices should be converted . Let's do little more...\\\\
\vdots\\

After some conversion, $T_2$ was converted to $R_1$ and $R_2$, $u_{h-1}(t+1)$ is converted to $u_1$ and $u_2$. Just group not so important variable sets to $S$. So our equation can be shown as 
\begin{align}\label{eq:4}
	y_h(t+1) = R_1u_1 + R_2u_2 + S
\end{align}
To using QP(Quadratic Programming), convert matrix with final form. It's last conversion, for now... :)



\begin{align*}
	y_h(t+1) &= Hu + f\\\\
	H = 
	\begin{bmatrix}
		R_1 & R_2 \\
		0 & 0
	\end{bmatrix}_{2h \times 2h}
	,\quad
	u &=
	\begin{bmatrix}
		u_1\\
		u_2
	\end{bmatrix}_{2h \times h}
	,\quad
	f = 
	\begin{bmatrix}
		S\\
		0
	\end{bmatrix}_{2h \times h}
\end{align*}





\subsection{When delay is 0}

Starting from \eqref{eq:3}, delay is 0 so 

\begin{equation} \label{eq:5}
\\
	y_p(t+1) = Cy_p(t) + Du_p(t+1) + Ex_p(t+1)
\end{equation}
\\

\begin{align*}
	y(t+n) &= AC^{n-1}y_p(t-p+1)\\
	&+ \sum_{i=2}^{n} AC^{n-i}Du_p(t+n-p+(i-n)) + B_1u_p(t+n-p+1)\\
	&+ \sum_{i=2}^{n} AC^{n-i}Ex_p(t+n-p+(i-n)) + B_2x_p(t+n-p+1)
\end{align*}
\\


\begin{align*}
	y_h(t+1) &= Tu_{p+h-1}(t-p+2) + Ux_{p+h-1}(t-p+2) + Vy_p(t-p+1)\\
	&= T_1u_{p-1}(t-p+2)+ T_2u_{h}(t+1) + Ux_{p+h-1}(t-p+2) + Vy_p(t-p+1)\\
	& = R_1u_1 + R_2u_2 + S
\end{align*}

\begin{align*}
	u &= u_1, u_2 = u_h(t+1)\\
	S &= T_1u_{p-1}(t-p+2) + Ux_{p+h-1}(t-p+2) + Vy_p(t-p+1)
\end{align*}


\clearpage









\section{Optimization}
\subsection{Introduction}

\hspace{\parindent}Objective of optimization in this problem is minimize energy usage($gas$) while keeping suitable indoor temperautre($T_{in}$). So first strategy is place $gas$ in objective function, and place $T_{in}$ in constraints.
Other important constraints are integer conditions. Since $u_1$ is signal and $u_2$ is boiler set point, $u_1$ is binary and $u_2$ is integer which range within 30 to 80.


$ARX$ model which predicts $T_{in}$ has such good prediction accuracy, but in signal off conditions, when boiler set point changes , output of model($T_{in}$) changes, which is not the expected result. So when building the $T_{in}$ model, boiler set point variables are replaced by multiplying boiler setpoint and signal. Through this process, model is becoming more robust. Therefore, during oprimization process, boiler set point must be 0 when signal is 0. This condition integrated in to inequality constraints. Occupancy must be considered when calculating $T_{in}$.

$ARX$ model which predicts $gas\:usage$ shows poor performance. It calculates non-zero values while signal is off(0). So signal is multiplying to $ARX$ model itself.

In this problem, LP will be used first and QP(Quadratic Programming) will be. In LP, $gas\:usage$ model is used as original version, and in QP, multiplying version will be used.














\end{document}